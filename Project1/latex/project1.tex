\documentclass[11pt]{article}

\usepackage[margin=1in]{geometry}
\usepackage{graphicx}
\usepackage{amsmath}
\usepackage{amsfonts}
\usepackage{amssymb}
\usepackage{hyperref}

\title{Lab 1 - ME155C/ECE147C}
\author{ColeG, RaaghavT, TienN}
\date{\today}

\begin{document}
\maketitle

% ---------------------------------------------------
% Abstract
% ---------------------------------------------------
\begin{abstract}
% [Report] Abstract:
% Summarize the content of the entire document in one or two paragraphs.
% Focus on key achievements.
%
% Example structure:
% - One paragraph explaining the overall problem or project goal
% - One paragraph highlighting main results and conclusions
\end{abstract}

% ---------------------------------------------------
% 1. Introduction
% ---------------------------------------------------
\section{Introduction}
% [Report] Introduction:
% Should generally cover:
% 1) Brief, self-contained description of the basic problem 
% 2) Summary and references to prior/related work 
% 3) Short paragraph on organization of this report.

% Provide context on the overall goal of your project:
% - E.g., “This lab project deals with … [controller design / system identification / etc.]”
% - Summarize what is known or what has been done before.
% - Outline the subsequent sections.

% Example:
%  - 1 paragraph describing the problem
%  - 1 paragraph describing prior related work
%  - 1 paragraph summarizing your contributions and how the rest of the report is organized

% Insert note about the planning aspect (mentioned in [Report] “Plan ahead …”)
\label{sec:intro}


% ---------------------------------------------------
% 2. System Identification
% ---------------------------------------------------
\section{System Identification}
\label{sec:systemID}

% [Report] System description:
% “Briefly describe the model of the process you are controlling, 
% explain meaning of the variables, their units, control input(s), measured output(s).”

\subsection{Process to be controlled}
% [Report] Provide a concise description of:
% - The physical system
% - Notation for variables, units
% - Control input(s) and measured output(s)
% For example, mention x1, x2, etc., masses, motor parameters, etc.
% Possibly include a figure

\subsection{Non-Parametric Identification}
% [Report] For the non-parametric approach:
% 1) Frequencies/amplitudes used and why
% 2) Plots with magnitude/phase
% 3) Comparison with parametric identification

\subsection{Parametric Identification}
% [Report] For the parametric approach:
% 1) Inputs used (square waves, chirp, etc.) and rationale
% 2) Plots showing input and output signals (representative examples)
% 3) Discussion of chosen model order, justification
% 4) Bode plots (magnitude and phase) of identified model
% 5) Comparison with non-parametric approach


% ---------------------------------------------------
% 3. Controller Design
% ---------------------------------------------------
\section{Controller Design}
\label{sec:control}

\subsection{Design Methodology}
% [Report] Summarize the design approach:
% - Requirements (overshoot, settling time, etc.)
% - The control method chosen (PID, state-feedback, etc.)
% - Justification (why these gains, design trade-offs, etc.)
% - Possibly references to classical design or modern control design approaches

\subsection{Simulation Results}
% [Report] Provide:
% - Step response plots for the closed-loop system 
%   (using the identified model AND the nominal model from ECE147A&B, if relevant)
% - Bode plots for the closed-loop
% - Any relevant discussion about stability, noise, etc.


% ---------------------------------------------------
% 4. Closed-loop Performance
% ---------------------------------------------------
\section{Closed-loop Testing}
\label{sec:closedLoop}

% [Report] Summarize the experiments performed in the lab with the actual hardware:
% 1) Step response data: include overshoot, rise time, settling time, max control input
% 2) Frequency response identification for the closed-loop (how you performed it, the results)
% 3) Discuss any discrepancies between theoretical and experimental results 
%    and possible reasons for them

\subsection{Step Response Experiments}
% [Report] Include a relevant plot/table with performance metrics (overshoot, rise time, etc.)

\subsection{Closed-loop Frequency Response}
% [Report] Show how you identified the closed-loop frequency response
% (method used, plots, comparison with simulated predictions, etc.)


% ---------------------------------------------------
% 5. Conclusions and Future Work
% ---------------------------------------------------
\section{Conclusions and Future Work}
% [Report] This normally covers:
% 1) Summary of main project achievements
% 2) Outline of what else could be done with more time or resources

% ---------------------------------------------------
% References
% ---------------------------------------------------
\begin{thebibliography}{9}
% [Report] Place any references here. 
% Example usage:
% \bibitem{somePaper} Author(s). "Title of Paper." Journal/Conference. Year.
\end{thebibliography}

\end{document}
